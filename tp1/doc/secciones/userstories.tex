\section{Requerimientos funcionales}

\newcommand{\BV}{Business Value}
\newcommand{\SP}{Story Points}
\newcommand{\US}[1]{\textbf{US #1}}
\newcommand{\UUSS}{User Stories}
\newcommand{\fixme}[1]{\large\textcolor{red}{#1}}

\subsection{User Stories - Introducción}

Originalmente habíamos definido el backlog pensando en interacciones de input y output que buscaba el usuario del simulador en el ministerio, pero dichas interacciones no tienen (o no es la mejor manera de entenderlo) valor para el usuario.
\\

Por lo tanto pensamos las User Stories partiendo de estrategias y funcionalidades que le interesarían a potenciales expertos del ministerio para que el simulador sea completo. De esta manera, descompusimos las estrategias del equipo de ingeniería en varias ramas y pensamos a qué área o experto del ministerio le interesaría poder usarlas. Esto nos permitió, además, diversificar nuestros potenciales stakeholders y tener una visión más general de los objetivos.
\\

Por ejemplo, a la gente de Geología, que conoce sobre las características físicas de las parcelas (i.e presión, profundidad, tipo de terreno y su resistencia a excavación), le interesaría que el simulador provea estrategias con distintos criterios de elección de parcelas a perforar para ver cuáles hacen mejor uso de las propiedades de los terrenos del yacimiento mejorando los resultados de la potencial explotación.\\

Además de lo que son las estrategias y criterios que el simulador provería a los 
distintos expertos, los mismos deben ser capaces de especificar y configurar el estado 
actual de la realidad, para que el simulador conozca la situación de mercado actual, 
las propiedades del yacimiento, los distintos modelos de RIGs que se pueden alquilar, etc. 

\subsection{User Stories}

Definimos las siguientes User Stories para un potencial Product Backlog de Scrum, con sus respectivos Story Points (SP) y Bussiness Value (BV). Para los Story Points usamos Fibonacci Scale.

\begin{center}
  \begin{tabular}{| r | p{13cm} | c | c | }
    \hline
    N° & Descripción & BV & SP\\  \hline

      % model - old
    1 & COMO Analista Económico QUIERO que el simulador permita establecer distintos propiedades de mercado actuales PARA poder hacer planificaciones y estimaciones monetarias a futuro realistas & 9 & 3\\ \hline

      % model - old
    2 & COMO Perito del yacimiento QUIERO poder especificar con suficiente detalle las propiedades físicas de mi yacimiento PARA que la producción de los pozos sea plausible en una potencial explotación real & 9 & 3\\ \hline

      % model
    3 & COMO Ing. en Perforaciones QUIERO poder fijar varios modelos alquilables de RIGs con distintas cualidades PARA poder evaluar los distintos desempeños de excavación según modelo. & 4 & 3 \\ \hline
      
      % model
    4 & COMO Ing. Químico QUIERO quiero poder definir distintos modelos de plantas separadoras PARA poder evaluar los distintos rendimientos de separación de producto & 4 & 2 \\ \hline

      % model
    5 & COMO Ing. Hidráulico QUIERO quiero poder definir modelos de tanques de almacenamiento de gas y agua PARA tener una mejor evaluación a la hora de elegir estrategias de almacenamiento de producto & 4 & 2 \\ \hline

    \end{tabular}
\end{center}

\begin{center}
  \begin{tabular}{| r | p{13cm} | c | c | }
    \hline
    N° & Descripción & BV & SP\\  \hline

      % strategy - old
    6 & COMO Geólogo QUIERO evaluar diferentes criterios de elección de parcelas a perforar PARA aprovechar las características físicas de los terrenos del yacimiento & 8 & 5\\  \hline

      % strategy - old
    7 & COMO Ing. en Perforaciones QUIERO poder elegir estrategias de uso y contratación de RIGs PARA evaluar el \textit{tradeoff} entre tiempos de excavación y costos & 6 & 8\\ \hline

      % strategy - old
    8 & COMO Ing. Químico QUIERO elegir tiempos y criterios de construcción de las plantas separadoras PARA conseguir resultados satisfactorios en el balance de costo, tiempo y producción & 5 & 8\\ \hline

      % strategy - old
    9 & COMO Ing. Hidráulico QUIERO planificar la construcción de los tanques de almacenamiento de agua PARA conseguir resultados satisfactorios en el balance de costo, tiempo y producción & 6 & 8\\ \hline

      % strategy - old
    10 & COMO Ing. Petroquímico QUIERO planificar la construcción de los tanques de almacenamiento de gas PARA conseguir resultados satisfactorios en el balance de costo, tiempo y producción & 6 & 8\\ \hline
      
      % strategy - old
    11 & COMO Ing. Petrolífero QUIERO administrar la habilitación de pozos PARA controlar la presión día a día, y por lo tanto el volumen extraído de producto & 9 & 8\\ \hline

      % strategy - old
    12 & COMO Ing. Petrolífero QUIERO evaluar varias estrategias automáticas de reinyección PARA poder aprovechar al máximo el yacimiento & 6 & 3\\ \hline

      % strategy
    13 & COMO Analista Económico QUIERO poder elegir criterios de compras de agua PARA tener más opciones a la hora de reinyectar & 5 & 3\\ \hline

      % strategy
    14 & COMO Analista Económico QUIERO poder elegir criterios de ventas de gas PARA ponderar \textit{tradeoff} entre el rédito económico de su venta y el almacenamiento para su utilización en reinyecciones & 5 & 3\\ \hline

      % strategy - old
    15 & COMO Experto de Finanzas QUIERO poder probar distintos criterios de corte de la simulación PARA saber cuándo conviene terminar de explotar el yacimiento & 4 & 1\\ \hline

      % old
    16 & COMO Ing. Petrolífero QUIERO poder conocer la producción de los pozos PARA poder calcular la ganancia de la explotación & 10 & 3\\ \hline
    
    %12 & COMO encargado de licitaciones QUIERO poder calcular el balance financiero de la explotacion PARA determinar el canon a cobrar & 9 & 3\\ \hline
  \end{tabular}
\end{center}

Al buscar expertos de cada área como interesados en funcionalidades, las user stories terminan tendiendo múltiples roles de usuario.

%\newpage
\subsection{Discusión de las User Stories}

Siguiendo con lo que decíamos antes, podemos caracterizar las 
distintas User Stories en dos grandes grupos según la funcionalidad que proveen.  
Por un lado están las que permiten al simulador representar el estado actual de la realidad, 
por ejemplo, qué modelos de plantas existen, cómo es el yacimiento a explotar, etc.
Por otro lado, están las que definen las distintas estrategias automáticas del 
simulador (para perforar parcelas, habilitar pozos, reinyectar, etc) que el equipo 
de ingeniería probará y combinará para evaluar la explotación 
del yacimiento y llegar a un veredicto. \\

Una posible alternativa que discutimos en su momento, fue la de separar en dos o más cada una de las stories que tratan de estrategias del equipo de ingeniería y dejar escrito, para cada una, qué estrategia específica (por ejemplo, una para un criterio de corte a los $N$ días y otra para un criterio de corte dado una cota en la composición del yacimiento) le interesaría que provea el simulador.

Decidimos preservarlas unificadas porque, en caso de ser dividas, surge el hecho de que no serían completamente independientes (una vez realizada una, la implementación de la otra reutiliza mucho de la anterior) haciendo que los Story Points se vuelvan poco significativos.
\\

A continuación explayamos un poco la idea de cada User Stories y las justificaciones que nos llevaron a esta asignación de business value y story points.

\begin{itemize}
  \item Asignamos un alto \BV{} a \US 1 y \US 2 (especificación del mercado y yacimiento). Nos parece fundamental a la hora de plantear el canon para una licitación tener bien modelado los aspectos particulares del simulador: las características físicas del yacimiento y el estado del mercado petrolífero. Sin ello todo lo que hace la simulación es de poca utilidad analítica. A su vez, su implementación requiere simplemente tomar input del usuario, y no incluye ningún componente complejo.

  \textbf{Nota:} Originalmente nuestro criterio para asignar BVs era pensar \emph{qué tan prioritario sería mostrarle al cliente un prototipo con dicha funcionalidad}. En cuyo caso, mostrarle una demo robusta en cuanto a estrategias pero con un contexto económico y físico fijo no resultaba tan negativo. Pero en una versión final sí era indispensable por lo mencionado anteriormente. Por lo que decidimos también adoptar un nuevo criterio de \emph{qué tan útil sigue siendo el simulador sin la funcionalidad}.

  \item Las \US{3}, \US{4} y \US{5} (especificación de RIGs, plantas y tanques) presentan bastantes similitudes. Todas son bastantes simples de implementar, siendo muy parecidas a las dos anteriores, pero además son menos parámetros los que terminan definiendo a estos elementos, por tanto tienen valores de \SP{} un poco reducidos (salvo  las plantas procesadoras por ser un poco más complejas). Respecto a los \BV{}, creemos que el simulador puede funcionar aceptablemente en una primera versión sin esta funcionalidad, por ejemplo si tiene fijos los modelos más usuales de RIGs, plantas y tanques del mercado. Por tanto, los \BV{} son algo bajos. 

  \item Para la \US{6} (elección de parcelas a perforar) creemos que el resultado de la elección de las parcelas va a ser algo muy concreto y visual para mostrar al cliente, por lo que tiene un alto business value. Además, es una base importante para el curso de la simulación. También tiene una cantidad moderada de story points porque requiere implementar al menos 2 estrategias.
  
  \item Para la \US{7} (estrategias de RIGs) sobre el \BV{}, si le presentamos al cliente un simulador donde los pozos se construyen automáticamente, sin costos y sin dificultades pero el resto de las estrategias y decisiones persisten el simulador sigue siendo un producto relativamente útil. 

	Las estrategias a implementar son complejas; hay muchas cosas a considerar, por ejemplo, la resistencia del suelo, consumo, mínima cantidad de días para alquilarlos, en qué momento, cuántos, etc. Por tanto tiene un valor alto de \SP{}.

	\item Hallamos similitudes entre la \US 8, \US 9 y \US{10} (construcción de plantas y tanques). Todas requieren implementar varias estrategias que tendrán en cuenta los parámetros de los elementos y su necesidad en función de la producción diaria. Por ello, consideramos un alto valor en \SP{}. Con respecto al \BV{}, consideramos que es una parte interesante del modelo pero al mismo tiempo se podría realizar una simulación previa asumiendo que las plantas procesadoras y los tanques no son el cuello de botella.

  \item Sobre la \US{11} (habilitación de pozos) creemos que es fundamental tener las estrategias de habilitación de pozos para el simulador (si esto fuera estático, se perdería gran control sobre la explotación del yacimiento y producción de los pozos día a día), por eso tiene uno de los \BV{} más altos. Tiene una dificultad de implementación moderadamente alta, al igual que otras funcionalidades con estrategias múltiples, y son muchas las variables a tener en cuenta en los mismos. Además es aquí donde debe implementarse la extracción de producto en sí. Por eso los \SP{} son relativamente altos. 

  \item Creemos que los criterios de reinyección para la \US{12} son sencillos de implementar, por ende sus bajos \SP{}. Al igual que otras funcionalidades, no es tan fundamental para tener una buena estimación del canon, dado que puede hacerse ``a mano'' en un principio. Es decir, cuando crean que es necesaria una reinyección, comenzar de nuevo una simulación con los niveles de composición y presión que crean necesarios.

  \item Para la \US{14} (compra de agua y venta de gas), son también relativamente sencillos de implementar, ambos tienen que ver un poco con los criterios de reinyección además. Sin embargo, son más sencillos que estos, incluso además se vuelven más fáciles si ya se los tienen implementados a los de reinyección. De modo que los \SP{} son bajos también, similares a los criterios de reinyección, pero un poco menos fundamentales que estos últimos, por tanto el \BV{} es un poco más bajo. 

  \item En la \US{15} (criterios de corte) también nos parece que la funcionalidad es algo que puede reemplazarse temporalmente mirando los logs para estimar cuándo convenía cortar, por tanto tiene un \BV{} bajo. También son muy fáciles de implementar, basta con chequear un par de valores dentro de la simulación, además su resultado es simple también (cortar o no cortar), por eso sus \SP{} son bajos. 

  \item La \US{16} tiene el \BV{} máximo. Lo primero que quiere ver el usuario al correr una simulación es el balance económico de la explotación. Con respecto a las \SP{}, se requiere llevar la cuenta día a día, pero consideramos que esto es algo bastante fácil de implementar.
\end{itemize}

Resulta interesante hacer énfasis en que las stories con mayor business value no son las que tratan de estrategias provistas sino sobre el contexto real del yacimiento (a nivel económico y físico) y la capacidad de analizar producción y remuneración. Esto se debe, a que desde la perspectiva del cliente, es prioritario que la simulación sea apropiada a la realidad para realizar estimaciones frente a qué tan óptimas son dichas estimaciones. Si el simulador fuera capaz de proveer estrategias óptimas aisladas de un contexto de su interés y sin proveer acceso al conteo de producción, esto no tendría absolutamente ninguna utilidad para el ministerio en su afán de estimar cánones.

\newpage
\subsection{Desarrollo de las User Stories más importantes}

En esta sección vamos a dar detalles de las descripciones, tareas, criterios de aceptación y estimación de RRHH para las User Stories que consideramos más importantes. Cabe aclarar que éstas podrían no necesariamente ser las que tengan una mayor relación de \( \frac{\BV{}}{\SP{}} \). \\

Si bien tomamos en cuenta este criterio, consideramos que hay otros factores que hacen a una User Story más interesante para ser desarrollada. Tampoco son necesariamente las primeras que deberían ser implementadas para el simulador. Creemos a pesar de todo, que la elección de estas tres para desarrollar da un buen panorama de cómo deberían ser el resto de las User Stories también, dado que la mayoría puede asemejarse bastante con alguna de estas (son representativas del todo de alguna manera). \\

Ponemos a continuación las 3 User Stories, para cada una indicamos su
\BV{} y \SP{} respectivamente. Las que elegimos fueron la \US{2}, \US{15} y \US{16}. \\

% criterio de corte
\begin{tcolorbox}
\textit{\US{15} COMO Experto de Finanzas QUIERO poder probar distintos criterios de corte de la simulación PARA saber cuándo conviene terminar de explotar el yacimiento [4/1]}\\

\textbf{Criterios de aceptación:}
\begin{itemize}
	\item Si el usuario elige el corte por límite de composición de petroleo, debe ver una simulación en la que todos los días el porcentaje se mantiene por encima del valor crítico, y el día final queda un porcentaje mayor a tal valor.
    \item Si el usuario elige el corte fijo al día $n$, debe observar una simulación con exactamente $n$ días desde el comienzo.
\end{itemize}

\textbf{Tareas:}
\begin{itemize}
	\item Definir e implementar el modelo para la clase abstracta de
    Criterio de Corte (20').
    \item Implementar la decisión de cortar por límite de composición. (1:20 hs)
    \item Implementar la decisión de cortar por día prefijado. (40')
\end{itemize}
\end{tcolorbox}

% calcular produccion
\begin{tcolorbox}
\textit{\US{16} COMO Ing. Petrolífero QUIERO poder conocer la producción de los pozos PARA poder calcular la ganancia de la explotación [10/3]}\\

\textbf{Criterios de aceptación:}
\begin{itemize}
	\item Debe quedar registrado en el log cuántos $m^3$ de producto se extrajeron día a día en cada pozo.
    \item Debe quedar registrado en el log cuántos $m^3$ de cada compuesto se van almacenando en tanques y cuánto de petróleo se extrajo.
    \item Debe quedar registrado en el log cuál es el valor total del emprendimiento hasta el día de la fecha.
\end{itemize}

\textbf{Tareas:}
\begin{itemize}
	\item Considerar distintas herramientas posibles para logging e añadir la más conveniente al proyecto. (2 hs)
  \item Agregar funcionalidad global que permita a cualquier objeto loggear la información que necesite. (30')
    \item Agregar los llamados de logging en la finalización de cada día de simulación. (1 hs)
    \item Agregar interfaces correspondientes para poder acceder a los datos necesarios a loggear. (1 hs)
\end{itemize}
\end{tcolorbox}

% caracteristicas yacimiento
\begin{tcolorbox}
\textit{\US{2} COMO Perito del yacimiento QUIERO poder especificar con suficiente detalle las propiedades físicas de mi yacimiento PARA que la producción de los pozos sea plausible en una potencial explotación real [9/3]}\\

\textbf{Criterios de aceptación:}
\begin{itemize}
	\item En base a mediciones reales en el yacimiento virgen que se va a licitar, el perito puede seleccionar la profundidad, resistencia a RIGS y presión inicial de cada parcela.

    \item También puede ajustar la composición petróleo/agua/gas y volumen inicial del yacimiento según dichas mediciones y la concentración crítica de petróleo para poder seguir funcionando.

    \item Podrá instanciar los coeficientes \textit{alfa} de las ecuaciones para ajustar el potencial volumen de extracción de cada pozo.
\end{itemize}

\textbf{Tareas:}
\begin{itemize}
	\item Definir e implementar una interfaz de usuario, ya sea por medio de texto o gráficos. (2:30 hs)
    \item Parsear y validar los datos ingresados por el usuario. (2 hs)
\end{itemize}
\end{tcolorbox}
